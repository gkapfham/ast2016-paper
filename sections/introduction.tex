%!TEX root=../ast2016.tex

\section{Introduction}
\label{sec:introduction}

% GMK NOTE: Here are some data points about the prevalence of queries of StackExchange about SQL databases

% select
% Count(Id)
% from Posts
% where
% (
% Tags like '%sql%'
% and Tags not like '%nosql%'
% or Tags like '%mysql%'
% or Tags like '%oracle%'
% or Tags like '%postgres%'
% or Tags like '%sqlite%'
% or Tags like '%derby%'
% )

% TOTAL = 876022 posts

% I have confirmed that screening out NoSQL using the above technique does work correctly and avoids the over-counting
% that would occur if the not like was not included

% Reference:
% http://data.stackexchange.com/stackoverflow/query/407453/prevalence-of-sql-databases

% GMK NOTE: Here are some data points about the prevalence of questions on StackExchange about NoSQL databases

% select
% Count(Id)
% from Posts
% where
% (
% Tags like '%nosql%'
% or Tags like '%mongodb%'
% or Tags like '%couchdb%'
% or Tags like '%cassandra%'
% or Tags like '%neo4j%'
% or Tags like '%hbase%'
% )

% TOTAL = 84314 posts

% Reference:
% http://data.stackexchange.com/stackoverflow/query/407458/prevalence-of-nosql-databases

%% Establish the importance of relational databases
\begin{sloppypar}
  Relational databases provide a reliable way to store and retrieve data, forming an critical component of a wide range of different software applications, from domains such as Internet browsers (e.g., Chrome\footnote{\small{\url{https://www.google.com/chrome/browser}}} and Firefox\footnote{\small{\url{http://www.mozilla.org/firefox}}}) to applications powering political campaigns~\cite{Butler2012}. Despite the recent wave of interest in ``NoSQL'' technologies, relational databases are still important relevant and popular in modern software application design, as evidenced by the $876,022$ questions on the popular StackExchange technical question and answer website tagged with labels devoted to the topic.\footnote{\small{\url{http://goo.gl/F3Tiax}}} % points to the URL mentioned in comments at the top of this file
\end{sloppypar}

%% Establish some key benefits and introduce the schema
Key benefits to using a relational database include the good performance of database management systems (DBMSs) \cite{Abrahami2015} --- such as \Postgres and \SQLite, two popular and free DBMSs --- and their reliability. Developers often prefer relational databases to other storage and retrieval systems due to the availability of a clear database {\it schema}, which specifies the data to be stored and how it is structured into tables, serving as easy-to-refer-to documentation\footnote{\small{\url{http://goo.gl/v03nUr}}}\hspace{-1ex}. % points to: http://stackoverflow.com/questions/3294755/with-the-recent-prevelance-of-nosql-databases-why-would-i-use-a-sql-database

%% Introduce integrity constraints
\begin{sloppypar}
A relational database schema further involves the definition of a series of {\it integrity constraints} that guard the validity and consistency of stored data. Integrity constraints ensure that certain data values are unique, through \PKCs and \UCs; maintain referential integrity with other data values, through \FKCs; are actually present, through \NNCs; and are subject to other arbitrary domain-specific conditions, through \CCs. Integrity constraints prevent invalid values being admitted into the database via SQL \INSERT statements, by causing the DBMS to reject the statement with an error. Integrity constraints encode key application logic, providing the broader software application with a last line of defense against malformed data entries. As such, and in accordance with industry advice \cite{DzoneDatabaseTesting}, they require thorough testing.
\end{sloppypar}

%% Introduce work to date
To this end, previous work has devised coverage criteria and automated test suite generation approaches for relational database schema integrity constraint testing \cite{Kapfhammer2013,McMinn2015}. Past work has also proposed mutation analysis techniques to seed faults in database schemas that simulate faults of commission and omission, which may be used to evaluate test suites for integrity constraints \cite{Kapfhammer2007,Wright2013}.
%
%% Introduce the problem
However, as with most forms of mutation --- including traditional program mutation analysis --- these techniques are costly to execute, since the test suite needs to be evaluated against every mutant to determine whether it is capable of exposing the seeded fault.

%% Removed sentence as we're not mentioning scalability anymore.
%Therefore, in common with program mutation, our mutation analysis methods do not scale well with the number of mutated schemas.

%% Introduce the solution and then transition into the listing of the important contributions

One cost incurred in evaluating relational database schema test suites is the overhead of communicating with the DBMS that hosts each of the mutant databases, which forms a significant component of the overall time needed to perform mutation analysis. In this paper, we propose to perform mutation analysis {\it virtually}, on a local model of the DBMS, rather than an actual running instance that incurs constant and costly interaction. Since different DBMSs interpret the SQL standard differently, and often have unique implementation ``quirks'', a model is required for each different DBMS. In this paper we utilize models for three popular and widely-used DBMSs: \HyperSQL, \Postgres and \SQLite. We empirically show how using virtual mutation instead of the \Original method can cut the costs of analysis by $51\%$ on average across all of the studied configurations. Thus, the contributions of this paper are:

\vspace{1mm} \noindent 1. Avoiding costly DBMS interactions, a new method for performing mutation analysis {\it virtually} using a
local model of three popular and widely-used DBMSs: \HyperSQL, \mbox{\Postgres and \SQLite.}

\vspace{1mm} \noindent 2. Demonstrating both the efficiency and the effectiveness of \mbox{virtual} mutation analysis, an empirical study
incorporating \numschemas relational database schemas and the three representative DBMSs.

%% Introduce the next section.
% No room?
%We begin the paper by introducing key background information.

