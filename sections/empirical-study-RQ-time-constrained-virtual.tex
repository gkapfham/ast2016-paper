% vim: ft=tex
%!TEX root=ast2016.tex

% GRAPHIC: This is the box and whisker plot that shows the mutation score for the two techniques
%!TEX root=../ast2016.tex

\begin{figure*}[t]
  \centering
  \includegraphics[scale=1.0]{graphics/graphic_bwplot_schema_mutationscore_vm_tcm.pdf}
  \caption{Box plot of the mutation score for the virtual and time-constrained mutation analysis techniques.}
  \label{fig:graphic_bwplot_schema_mutationscore_vm_tcm}

  % NOTE: There is no need to duplicate the content that describes the meaning of the box plot's elements. So, this
  % subcaption explains more about the configuration of the experiment that produced this graph, which is useful.

  {\small \justifying{ \noindent The meaning of this box plot's elements is the same as the meaning of those described
      in the subcaption of Figure~\ref{fig:graphic_bwplot_schema_analysistime_org_vm}. This plot shows the variation in
      the mutation score for all of the chosen relational schemas and the three database management systems, using test
      suites from thirty runs of the search-based test data generation method developed by McMinn et al.\
      \cite{McMinn2015}}. The time-constrained approach was allowed to analyze mutants for as long as the virtual mutation
      method would have run for the corresponding schema. \par}

\end{figure*}


% GRAPHIC: This is the bar chart of the number of mutants that each technique ran during mutation analysis
% NOTE: This graph was removed due to space constraints, it can be summarized in the text, I think.
% %!TEX root=ast2016.tex

\begin{figure*}[t]
  \vspace*{-.25em}
  \centering
  \includegraphics[scale=1.0]{graphics/graphic_barplot_schema_mutantcount_vm_tcm.pdf}
  % \vspace*{-.5em}
  \caption{Bar plot of the mutant count for both the selective and virtual mutation analysis techniques.}
  \label{fig:graphic_barplot_schema_mutantcount_vm_tcm}

  {\small \justifying{ \noindent In this plot the height of the bar corresponds to the number of mutants subject to
      analysis by the selective and virtual methods, reported for all of the schemas and the three DBMSs. Since
      the selective technique employs randomness to pick mutants that can be run within a specified time limit,
      the height of a dark grey bar is the average across a total of thirty runs; virtual mutation is deterministic and
  thus the height of the light grey bar is a direct count. } \par}

  \vspace*{-1em}

\end{figure*}


\inlineheading{Virtual and Time-Constrained Mutation} Since the experiments revealed that \vma~is faster than the \Original~one in $22$ out of the $27$ studied configurations --- and competitive with the DBMS-based method in the other $5$ --- it is useful to ascertain whether the presented technique might yield more accurate mutation scores in some circumstances. To this end, Figure~\ref{fig:graphic_bwplot_schema_mutationscore_vm_tcm} presents the mutation score of both the virtual approach and a time-limited analysis in which \Original~randomly analyses mutants for as long as virtual. These box plots show that the time-constrained technique results in mutation scores that are often highly variable. This result can be attributed to randomness inherent in running mutation analysis under a strict time limit that will not permit the examination of every mutant. For instance, the noticeable variability in mutation score when the Person schema is run on \Postgres~is due to the possibility of not finishing the analysis of the first mutant.

Bearing in mind that the virtual method produces mutation scores that are always equal to those achieved by the standard technique, it is also important to note that time-constrained mutation analysis leads to overly high mutation scores.





