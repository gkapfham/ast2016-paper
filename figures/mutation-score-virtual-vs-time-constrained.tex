%!TEX root=../ast2016.tex

\begin{figure*}[t]
  \centering
  \includegraphics[scale=1.0]{graphics/graphic_bwplot_schema_mutationscore_vm_tcm.pdf}
  \caption{Box plot of the mutation score for the virtual and time-constrained mutation analysis techniques.}

  % Details about the box plot from the R documentation:

  % The lower and upper "hinges" correspond to the first and third quartiles (the 25th and 75th percentiles). This differs
  % slightly from the method used by the boxplot function, and may be apparent with small samples. See boxplot.stats for for
  % more information on how hinge positions are calculated for boxplot.

  % The upper whisker extends from the hinge to the highest value that is within 1.5 * IQR of the hinge, where IQR is the
  % inter-quartile range, or distance between the first and third quartiles. The lower whisker extends from the hinge to
  % the lowest value within 1.5 * IQR of the hinge. Data beyond the end of the whiskers are outliers and plotted as points
  % (as specified by Tukey).

  {\small \justifying{ \noindent The meaning of this box plot's elements is the same as the meaning of those described
      in the subcaption of Figure~\ref{fig:graphic_bwplot_schema_analysistime_org_vm}. This plot shows the variation in
      the mutation score for all of the chosen relational schemas and the three database management systems, using test
      suites from thirty runs of the search-based test data generation method developed by McMinn et al.\
  \cite{McMinn2015}}. The time-constrained approach was allowed to analyze mutants for as long as the virtual mutation
method would have run for the corresponding schema. \par}

\end{figure*}
